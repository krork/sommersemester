\documentclass{article}
\usepackage[utf8]{inputenc}
\usepackage{multicol}
\usepackage{amsmath}
\usepackage{float}
\usepackage{epsfig,graphicx}
\usepackage{xcolor,import}
\usepackage{subcaption}
\usepackage[font=small,labelfont=bf]{caption}
\usepackage{siunitx}
\usepackage[german]{babel}
\usepackage{textcomp}
\usepackage{mathtools}

\begin{document}


\thispagestyle{empty}
			\begin{center}
			\Large{Fakultät für Physik}\\
			\end{center}
\begin{verbatim}


\end{verbatim}
							%Eintrag des Wintersemesters
			\begin{center}
			\textbf{\LARGE SOMMERSEMESTER 2015}
			\end{center}
\begin{verbatim}


\end{verbatim}
			\begin{center}
			\textbf{\LARGE{Physikalisches Praktikum II}}
			\end{center}
\begin{verbatim}




\end{verbatim}

			\begin{center}
			\textbf{\LARGE{PROTOKOLL}}
			\end{center}
			
\begin{verbatim}





\end{verbatim}

			\begin{flushleft}
			\textbf{\Large{Experiment (Nr., Titel):}}\\
							%Experiment Nr. und Titel statt den Punkten eintragen
			\LARGE{}	
			\end{flushleft}

\begin{verbatim}

\end{verbatim}	
							%Eintragen des Abgabedatums, oder des Erstelldatums des Protokolls
			\begin{flushleft}
			\textbf{\Large{Datum:}} \Large{}
			\end{flushleft}
			
\begin{verbatim}
\end{verbatim}
							%Namen der Protokollschreiber
		\begin{flushleft}
			\textbf{\Large{Bachleitner Veronika, Grafendorfer Erik}} 
			\end{flushleft}

\begin{verbatim}


\end{verbatim}
							%Kurstag und Gruppennummer, zb. Fr/5
			\begin{flushleft}
			\textbf{\Large{Kurstag/Gruppe:}} \Large{FR/1}
			\end{flushleft}

\begin{verbatim}






\end{verbatim}
							%Name des Betreuers, das Praktikum betreute.
			\begin{flushleft}
			\LARGE{\textbf{Betreuer:\Large{ }}}		
			\end{flushleft}
			
\section{Aufgabenstellung}

\section{Theorie}
\subsection{}

\section{Aufbau}

\section{Durchführung}

\section{Ergebnisse}
\subsection*{Auflösungsvermögen eines Gitter}
%Berechnen Sie das Auflösungsvermögen A aus den Angaben.
$\lambda=576.96nm$\\
$\lambda=579.07nm$\\
Das Auflösungsvermögen kann berechnet werden mittels:
$$A=\frac{\lambda}{\Delta\lambda}=\frac{576.96}{2.11}=273.44$$
wobei wir $\Delta\lambda$ als die Differenz zwischen den beiden Wellenlängen genommen haben.\\
\\
%Leiten Sie die Formel zur Berechnung der Gitterkonstante aus dem Auflösungsvermögen und den Messgrößen her.
Das Auflösungsvermögen ist außerdem gegeben durch $A=n\cdot N$. Über die Formel $B=a\cdot N \Rightarrow N=\frac{B}{a}$ können wir also das Auflösungsvermögen und die Gitterkonstante in Beziehung stellen:
$$A=n\frac{B}{a}$$
Formen wir um, können wir aus den gemessenen Spaltbreiten zur jeweiligen Ordnung die Gitterkonstante berechnen:
$$a=n\frac{B}{A}$$

%Bestimmen Sie die Gitterkonstante und geben Sie diese mit Fehler an.
\begin{center}
\begin{tabular}{|r|l|l|}
n & B (mm, $\pm 0.01$) & a (mm)\\
\hline
R1 & 5.11 & 0.019\\
R2 & 1.85 & 0.013\\
R3 & 1.02 & 0.011\\
L1 & 5.15 & 0.019\\
L2 & 2.29 & 0.017\\
L3 & 1.78 & 0.019\\
%\caption{Spaltbreiten zu den jeweiligen Ordnungen. R: Rechts, L: Links. }
\end{tabular}

\end{center}
\vspace{1cm}
Aus den jeweils berechneten Werten für $a$ berechnen wir einen Mittelwert mit Standardabweichung:
$$\bar{a}=(0.016 \pm 0.004)mm$$

\subsection{Spektrometrie}
\subsection{Michelson-Interferometer}
\begin{table}
\caption{Spiegelverschiebung für N= 100 Umdrehungen}
\begin{tabular}{c}

$(29.5\pm 0.5 \mu m)$\\
$(32.5\pm 0.5 \mu m)$\\
$(32.0\pm 0.5 \mu m)$\\
\end{tabular}
\end{table}
\section{Diskussion}		
																								
\end{document}
