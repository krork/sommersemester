\documentclass[12pt,a4paper,twopage]{article}
\usepackage[utf8]{inputenc}
\usepackage[a4paper,margin=1cm,footskip=.5cm]{geometry}
\usepackage{multicol}
\usepackage{amsmath}
\usepackage{float}
\usepackage{epsfig,graphicx}
\usepackage{xcolor,import}
\usepackage{subcaption}
\usepackage[font=small,labelfont=bf]{caption}
\usepackage{siunitx}
\usepackage[german]{babel}
\usepackage{textcomp}
\usepackage{mathtools}
\linespread{1.1}
\usepackage{parskip}
\setlength{\parindent}{12pt}

\begin{document}


\thispagestyle{empty}
			\begin{center}
			\Large{Fakultät für Physik}\\
			\end{center}
\begin{verbatim}


\end{verbatim}
							%Eintrag des Wintersemesters
			\begin{center}
			\textbf{\LARGE SOMMERSEMESTER 2015}
			\end{center}
\begin{verbatim}


\end{verbatim}
			\begin{center}
			\textbf{\LARGE{Physikalisches Praktikum II}}
			\end{center}
\begin{verbatim}




\end{verbatim}

			\begin{center}
			\textbf{\LARGE{PROTOKOLL}}
			\end{center}
			
\begin{verbatim}





\end{verbatim}

			\begin{flushleft}
			\textbf{\Large{Experiment (Nr., Titel):}}\\
			PS9, Heißluftmotor - Stirlingprozess
							%Experiment Nr. und Titel statt den Punkten eintragen
			\LARGE{}	
			\end{flushleft}

\begin{verbatim}

\end{verbatim}	
							%Eintragen des Abgabedatums, oder des Erstelldatums des Protokolls
			\begin{flushleft}
			\textbf{\Large{Datum:}} \Large{29.5.2015}
			\end{flushleft}
			
\begin{verbatim}
\end{verbatim}
							%Namen der Protokollschreiber
		\begin{flushleft}
			\textbf{\Large{Bachleitner Veronika, Grafendorfer Erik}} 
			\end{flushleft}

\begin{verbatim}


\end{verbatim}
							%Kurstag und Gruppennummer, zb. Fr/5
			\begin{flushleft}
			\textbf{\Large{Kurstag/Gruppe:}} \Large{FR/1}
			\end{flushleft}

\begin{verbatim}






\end{verbatim}
							%Name des Betreuers, das Praktikum betreute.
			\begin{flushleft}
			\LARGE{\textbf{Betreuer:\Large{ }}}		
			\end{flushleft}
			
\section{Aufgabenstellung}

\section{Theorie}
\subsection{Allgemeine Grundlagen}
Es gibt thermodynamische Kreisprozesse, bei denen durch verschiedene Zustandsänderungen einem System Wärme zugeführt oder entnommen wird. Solche Zustandsänderungen sind isotherm (Temperatur konstant), adiabatisch (Wärmemenge konstant) und isochor (Volumen konstant).
\subsubsection{Carnot Prozess}
Eine Carnot-Maschine benützt eine Arbeitssubstanz, mit der sie einen quasistatischen Kreisprozess ausführt:\\
\\
1. Isotherme Zustandsänderung (Expansion). Dabei nimmt as Arbeitsgas die Wärmemenge $Q_{zu}$ auf.\\
2. Adiabatische Zustandsänderung (Expansion). Dabei hat das Arbeitsgas am Ende die Temperatur $T_2$.\\
3. Isothemre Zustandsänderung (Kompression). Dabei gibt as Arbeitsgas die Wärmemenge $Q_{ab}$ ab.\\
4. Adiabatische Zustandsänderung (Kompression). Dabei hat das Arbeitsgas am Ende die Temperatur $T_1$.\\
\\
Technisch ist der Carnot-Prozess jedoch nicht durchführbar, da einerseits die direkte Abfolge dieser Zustandsänderungen nicht realisierbar und andererseits die Prozesse in den meisten Wärmekraftmaschinen nicht reversibel sind.\\
\\
Der Wirkungsgrad ist gegeben durch:
$$\eta=\frac{A}{Q_{zu}}=\frac{Q_{zu}-Q_{ab}}{Q_{zu}}=1-\frac{Q_{ab}}{Q_{zu}}$$
wo $A$ die Arbeit und $Q_{zu}$ bzw. $Q_{ab}$ die zugeführte bzw. abgeführte Wärmemenge sind.\\
\\
Der Stirling-Kreisprozess besteht aus isothermen und isochoren Zustandsänderungen:\\
\\
1. Isotherme Zustandsänderung (Expansion).\\
2. Isochore Zustandsänderung (Expansion).\\
3. Isotherme Zustandsänderung (Kompression).\\
4. Isochore Zustandsänderung (Kompression).\\
\\
Verwendet man die ideale Gasgleichung ($pV=NkT$) sowie die Formeln für die Zustandsänderungen erhält man für den (idealen) Wirkungsgrad:
$$\eta=\frac{T_1-T_2}{T_1}=1-\frac{T_2}{T_1}$$
Der reale Wirkungsgrad ist dann gegeben durch
$$\eta_{real}=\frac{P_{Motor}}{P_{zu}}$$
wo $P$ die Leistung (Motor bzw. zugeführt) ist. Die Leistung kann in unserem ersten Versuch auch über folgendes Skalarprodukt bestimmt werden:
$$P=\vec{M}\cdot\vec{\omega}=\vec{F}\times\vec{r}\cdot\vec{\omega}$$
wo $\vec{M}$ das Bremsdrehmoment, $vec{F}$ die Kraft, $\vec{r}$ der Radius und $\vec{\omega}$ die Winkelgeschwindigkeit ist.\\
\\
Bei der \textit{Stirling-Maschine als Kältemaschine} wird im Prinzip ein Kreisprozess wie der oben beschrieben durchgeführt. Da es sich allerdings um eine Kältemaschine handeln soll, wird der Kreispozess jedoch gegen den Uhrzeigersinn durchlaufen.

\section{Aufbau}
\subsection{Wärmekraftmaschine}
\subsection{Kältemaschine}

\section{Durchführung}

\subsection{Wärmekraftmaschine}
\subsection{Kältemaschine}

\section{Ergebnisse}
\subsection{Wärmekraftmaschine}
\subsection{Kältemaschine}

\section{Diskussion}
\subsection{Wärmekraftmaschine}
\subsection{Kältemaschine}
																								
\end{document}
