\documentclass[12pt,a4paper,twopage]{article}
\usepackage[utf8]{inputenc}
\usepackage{geometry}
\geometry{a4paper,left=30mm,right=30mm, top=3cm, bottom=30mm} 
\usepackage{multicol}
\usepackage{amsmath}
\usepackage{float}
\usepackage{epsfig,graphicx}
\usepackage{xcolor,import}
\usepackage{subcaption}
\usepackage[font=small,labelfont=bf]{caption}
\usepackage{siunitx}
\usepackage[german]{babel}
\usepackage{textcomp}
\usepackage{mathtools}
\linespread{1.1}
\usepackage{parskip}
%\setlength{\parskip}{14pt}
\setlength{\parindent}{12pt}

\begin{document}

\begin{verbatim}


\end{verbatim}

\thispagestyle{empty}
			\begin{center}
			\Large{Fakultät für Physik}\\
			\end{center}
\begin{verbatim}


\end{verbatim}
							%Eintrag des Wintersemesters
			\begin{center}
			\textbf{\LARGE SOMMERSEMESTER 2015}
			\end{center}
\begin{verbatim}


\end{verbatim}
			\begin{center}
			\textbf{\LARGE{Physikalisches Praktikum II}}
			\end{center}
\begin{verbatim}




\end{verbatim}

			\begin{center}
			\textbf{\LARGE{PROTOKOLL}}
			\end{center}
			
\begin{verbatim}





\end{verbatim}

			\begin{flushleft}
			\textbf{\Large{Experiment Nr. 7, Halbleiter 1: Dioden, Gleichrichtung}}\\
							%Experiment Nr. und Titel statt den Punkten eintragen
			\LARGE{}	
			\end{flushleft}

\begin{verbatim}

\end{verbatim}	
							%Eintragen des Abgabedatums, oder des Erstelldatums des Protokolls
			\begin{flushleft}
			\textbf{\Large{Datum:}} \Large{15.5.2015}
			\end{flushleft}
			
\begin{verbatim}
\end{verbatim}
							%Namen der Protokollschreiber
		\begin{flushleft}
			\textbf{\Large{Bachleitner Veronika, Grafendorfer Erik}} 
			\end{flushleft}

\begin{verbatim}


\end{verbatim}
							%Kurstag und Gruppennummer, zb. Fr/5
			\begin{flushleft}
			\textbf{\Large{Kurstag/Gruppe:}} \Large{FR/1}
			\end{flushleft}

\begin{verbatim}

\end{verbatim}
							%Name des Betreuers, das Praktikum betreute.
			\begin{flushleft}
			\LARGE{\textbf{Betreuer:\Large{}}}		
			\end{flushleft}
\newpage
\begin{verbatim}


\end{verbatim}
			
\section{Aufgabenstellung}
\section{Theorie}
Schockley-Gleichung:
\begin{equation}
I_D=I_S\cdot\left(\exp^{\frac{U_A e \beta}{n}}-1\right)
\label{shockley}
\end{equation}
wo $\beta e=26\si{mV}$ bei Raumtemperatur ($T=300$K)
\subsection{}
\section{Aufbau}
\section{Durchführung}
\section{Ergebnisse}
\subsection{Eigenschaften verschiedener Dioden}
%Strom-Spannungs-Kennlinie von R

%Kennlinie einer Si-Diode; Fit mit Shockley-Gleichung in Durchlass-Richtung

%Kennlinien 4 verschiedener Zener-Dioden; Zenerspannung bestimmen (I=-1mA)

%Kennlinien von 4 LEDs

%Optisches Spektrum der 4 LEDs mit einem automatischen Spektrometer; Kommentar zum Zusammenhang des Spektrums mit dem Kennlinienverlauf

\subsection{Diode als Gleichrichter}

\section{Diskussion}															
\end{document}
