\documentclass{article}
\usepackage[utf8]{inputenc}
\usepackage{multicol}
\usepackage{amsmath}
\usepackage{float}
\usepackage{epsfig,graphicx}
\usepackage{xcolor,import}
\usepackage{caption}
\usepackage{subcaption}
\usepackage[font=small,labelfont=bf]{caption}
\usepackage{siunitx}
\usepackage[german]{babel}
\usepackage{textcomp}
\usepackage{mathtools}
\usepackage{subcaption}
\usepackage{cleveref}

\begin{document}


\thispagestyle{empty}
			\begin{center}
			\Large{Fakultät für Physik}\\
			\end{center}
\begin{verbatim}


\end{verbatim}
							%Eintrag des Wintersemesters
			\begin{center}
			\textbf{\LARGE SOMMERSEMESTER 2015}
			\end{center}
\begin{verbatim}


\end{verbatim}
			\begin{center}
			\textbf{\LARGE{Physikalisches Praktikum II}}
			\end{center}
\begin{verbatim}




\end{verbatim}

			\begin{center}
			\textbf{\LARGE{PROTOKOLL}}
			\end{center}
			
\begin{verbatim}





\end{verbatim}

			\begin{flushleft}
			\textbf{\Large{Experiment Nr.5: Polarisation}}\\
							%Experiment Nr. und Titel statt den Punkten eintragen
			\LARGE{}	
			\end{flushleft}

\begin{verbatim}

\end{verbatim}	
							%Eintragen des Abgabedatums, oder des Erstelldatums des Protokolls
			\begin{flushleft}
			\textbf{\Large{Datum:}} \Large{24.04.2015}
			\end{flushleft}
			
\begin{verbatim}
\end{verbatim}
							%Namen der Protokollschreiber
		\begin{flushleft}
			\textbf{\Large{Bachleitner Veronika, Grafendorfer Erik}} 
			\end{flushleft}

\begin{verbatim}


\end{verbatim}
							%Kurstag und Gruppennummer, zb. Fr/5
			\begin{flushleft}
			\textbf{\Large{Kurstag/Gruppe:}} \Large{FR/1}
			\end{flushleft}

\begin{verbatim}






\end{verbatim}
							%Name des Betreuers, das Praktikum betreute.
			\begin{flushleft}
			\LARGE{\textbf{Betreuer:\Large{ KLEPP }}}		
			\end{flushleft}
			
\section{Aufgabenstellung}

\section{Theorie}
\subsection{Brewster Winkel}
\subsection{Spannungsoptik}
\subsection{Optische Aktivität}

\section{Aufbau}
\subsection{Brewster Winkel}
\subsection{Spannungsoptik}
\subsection{Optische Aktivität}

\section{Durchführung}
\subsection{Brewster Winkel}
\subsection{Spannungsoptik}
\subsection{Optische Aktivität}

\section{Ergebnisse}
\subsection{Brewster Winkel}
Aus dem gemessenen Winkel $\delta$ ergibt sich der Einfallswinkel $\alpha=\frac{180-\delta}{2}$\\
Der Winkel $\delta=180-2\alpha_1$
Ohne Platte 47\% bei 3mA bei -0°20'

\begin{table}[H]
\begin{center}
\begin{tabular}{|c|c|c|c|}
\hline
$\alpha$ ($^\circ$) & $\delta$ ($^\circ$) & $I_1$ ($mm$) & $I_2$ ($mm$)\\
\hline
35 & 110 & 0.114 & 0.045\\
40 & 100 & 0.132 & 0.037\\
45 & 90 & 0.162 & 0.027\\
50 & 80 & 0.192 & 0.019\\
55 & 70 & 0.228 & 0.0132\\
60 & 60 & 0.285 & 0.0183\\
65 & 50 & 0.370 & 0.041\\
\hline
\end{tabular}
\end{center}
\end{table}

%\begin{table}
%\begin{center}
%\begin{tabular}{|c|c|c|c|}
%\hline
%$\alpha$ & $\delta$ & $I_1$ & $I_2$\\
%\hline
%35 & 110 & 38 bei 0.3 & 45 bei 0.1 -> 0.045
%40 & 100 & 44 & 37 bei 0.1 -> 0.037
%45 & 90 & 54 & 27 
%50 & 80 & 64 & 19
%55 & 70 & 76 & 44 bei 0.03 -> 0.0044*3=0.0132
%60 & 60 & 95 und 29 bei 1 & 61 bei 0.03 bzw. 20 bei 0.1
%65 & 50 & 37 bei 1 & 41 bei 0.1
%\hline
%\end{tabular}
%\end{center}
%\end{table}


\subsection{Spannungsoptik}
Durchmesser $a=7.5cm$, also Fläche: $A=(a/2)^2\pi=441.7mm^2$\\
Dicke der Probe: $d=10.3mm$\\
Querschnittsfläche: $10.3mm^2=106.09mm^2$\\
$1 bar = 0.1 N/mm^2$\\
$S=\frac{\sigma d}{\delta}$\\
$\lambda=590nm$\\
$C_=\frac{\lambda}{S}$\\
\\
\begin{table}[H]
\begin{center}
\begin{tabular}{|c|c|c|c|c|r|}
\hline
$n$ & $p$ ($\si{bar}$, $\pm 0.5 bar$) & $p$ ($N/mm^2$, $\pm 0.05N/mm^2$) & $F$ ($N$) & $\sigma$ ($N/mm^2$) & $S$ ($N/mm$)\\
\hline
1 & 2.0 & 0.20 & 883.4 & 8.32852 & $85.7838 \pm 24$\\
2 & 5.5 & 0.55 & 2429.83 & 22.9034 & $117.953 \pm 12$\\
3 & 9.0 & 0.90 & 3976.1 & 37.3784 & $128.676 \pm 8.1$\\
4 & 12.0 & 1.20 & 5301.4 & 49.9711 & $128.676 \pm 6.0$\\
5 & 15.5 & 1.55 & 6847.69 & 64.546 & $132.965 \pm 4.8$\\
\hline
\end{tabular}
\end{center}
\end{table}
\vspace{0.5mm}
Für alle 5 Ordnungen:

$13°9'$
$$S_5=(134 \pm 21 )\cdot 10^3 N/mm \Rightarrow C_5=(4.40 \pm 0.71) \cdot 10^{-9}mm^2/N$$
Für Ordnungen 2 bis 5:
$$S_4=(143.1 \pm 7.1 )\cdot 10^3 N/mm \Rightarrow C_4=(4.12 \pm 0.21) \cdot 10^{-9}mm^2/N$$

\subsection{Optische Aktivität}
2 der Quarzproben drehten im mathematisch positiven Sinn, eine im mathematisch negativen Sinn.\\


\section{Diskussion}
\subsection{Brewster Winkel}
\subsection{Spannungsoptik}

Der Mittelwert aller fünf Messwerte ist $S=134 \pm 21 )\cdot 10^3 N/mm$. Vernachlässigen wir allerdings den Wert bei der 1. Ordnung erhalten wir $S=143.1 \pm 7.1 )\cdot 10^3 N/mm$; dieser Wert liegt sehr gut innerhalb der 1$\sigma$ Vertrauensbereiche der anderen Messwerte! Darum gehen wir davon aus dass die Messung bei der 1. Ordnung nicht genau war, eine Feststellung, die weiter davon bestärkt wird dass in einer zweiten Messung ein Wert von $2.5 \si{bar}$ gemessen wurde.\\
Bedenken wir zusätzlich, dass wir eine Unsicherheit von $0.5 \si{bar}$ für die Messung des Drucks annehmen, bedeutet das also, dass der Wert der 1. Ordnung eigentlich ($2.0 \pm 1.0)\si{bar}$ beträgt. Das ist bereits eine relative Unsicherheit von 50\%, wodurch auch dieser Messwert innerhalb der Unsicherheitsgrenzen liegt. 

\subsection{Optische Aktivität}
																					
\end{document}
