\documentclass{article}
\usepackage[utf8]{inputenc}
\usepackage{multicol}
\usepackage{amsmath}
\usepackage{float}
\usepackage{epsfig,graphicx}
\usepackage{xcolor,import}
\usepackage{subcaption}
\usepackage[font=small,labelfont=bf]{caption}
\usepackage{siunitx}
\usepackage[german]{babel}
\usepackage{textcomp}
\usepackage{mathtools}

\begin{document}


\thispagestyle{empty}
			\begin{center}
			\Large{Fakultät für Physik}\\
			\end{center}
\begin{verbatim}


\end{verbatim}
							%Eintrag des Wintersemesters
			\begin{center}
			\textbf{\LARGE SOMMERSEMESTER 2015}
			\end{center}
\begin{verbatim}


\end{verbatim}
			\begin{center}
			\textbf{\LARGE{Physikalisches Praktikum II}}
			\end{center}
\begin{verbatim}




\end{verbatim}

			\begin{center}
			\textbf{\LARGE{PROTOKOLL}}
			\end{center}
			
\begin{verbatim}





\end{verbatim}

			\begin{flushleft}
			\textbf{\Large{Experiment 2: Elektrische Schwingungen}}\\
							%Experiment Nr. und Titel statt den Punkten eintragen
			\LARGE{}	
			\end{flushleft}

\begin{verbatim}

\end{verbatim}	
							%Eintragen des Abgabedatums, oder des Erstelldatums des Protokolls
			\begin{flushleft}
			\textbf{\Large{Datum:}} \Large{20.03.2015}
			\end{flushleft}
			
\begin{verbatim}
\end{verbatim}
							%Namen der Protokollschreiber
		\begin{flushleft}
			\textbf{\Large{Bachleitner Veronika, Grafendorfer Erik}} 
			\end{flushleft}

\begin{verbatim}


\end{verbatim}
							%Kurstag und Gruppennummer, zb. Fr/5
			\begin{flushleft}
			\textbf{\Large{Kurstag/Gruppe:}} \Large{FR/1}
			\end{flushleft}

\begin{verbatim}






\end{verbatim}
							%Name des Betreuers, das Praktikum betreute.
			\begin{flushleft}
			\LARGE{\textbf{Betreuer:\Large{ }}}		
			\end{flushleft}
			
\begin{figure}
\includegraphics[angle=-90]{}
\end{figure}
\section{Aufgabenstellung}
An Kondensatoren und Spulen in verschiedenen Serienschaltungen sollen die Effekte eines Schwingkreises untersucht werden, wie das Abklingen, das durch Energieverlust an Leitungswärme entsteht, oder die Effekte, die durch anschließen eines zweiten Schwingkreises entstehen: Erzwungene Schwingungen und Schwebung.
\section{Theorie}
Ein elektrischer Schwingkreis aus einem Kondensator mit einer Kapazität C und einer Spule mit einer Induktivität L in einem Stromkreis mit der Spannung U lässt sich mit der gleichen Differentialgleichung beschreiben wie ein mechanischer Oszillator:

\begin{equation}
\label{HarmOsz}
L\frac{d^2q}{dt^2}+RI-\frac{q}{C}=0
\end{equation}
\section{Aufbau}

\section{Durchführung}

\section{Ergebnisse}
\subsection{Freie Schwingungen}
\subsection{Erzwungene Schwingungen}
\subsection{Gekoppelte Schwingkreise}
\section{Diskussion}		
																								
\end{document}
