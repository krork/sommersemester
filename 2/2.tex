\documentclass{article}
\usepackage[utf8]{inputenc}
\usepackage{multicol}
\usepackage{amsmath}
\usepackage{float}
\usepackage{epsfig,graphicx}
\usepackage{xcolor,import}
\usepackage{subcaption}
\usepackage[font=small,labelfont=bf]{caption}
\usepackage{siunitx}
\usepackage[german]{babel}
\usepackage{textcomp}
\usepackage{mathtools}

\begin{document}


\thispagestyle{empty}
			\begin{center}
			\Large{Fakultät für Physik}\\
			\end{center}
\begin{verbatim}


\end{verbatim}
							%Eintrag des Wintersemesters
			\begin{center}
			\textbf{\LARGE SOMMERSEMESTER 2015}
			\end{center}
\begin{verbatim}


\end{verbatim}
			\begin{center}
			\textbf{\LARGE{Physikalisches Praktikum II}}
			\end{center}
\begin{verbatim}




\end{verbatim}

			\begin{center}
			\textbf{\LARGE{PROTOKOLL}}
			\end{center}
			
\begin{verbatim}





\end{verbatim}

			\begin{flushleft}
			\textbf{\Large{Experiment 2: Elektrische Schwingungen}}\\
							%Experiment Nr. und Titel statt den Punkten eintragen
			\LARGE{}	
			\end{flushleft}

\begin{verbatim}

\end{verbatim}	
							%Eintragen des Abgabedatums, oder des Erstelldatums des Protokolls
			\begin{flushleft}
			\textbf{\Large{Datum:}} \Large{20.03.2015}
			\end{flushleft}
			
\begin{verbatim}
\end{verbatim}
							%Namen der Protokollschreiber
		\begin{flushleft}
			\textbf{\Large{Grafendorfer Erik}} 
			\end{flushleft}

\begin{verbatim}


\end{verbatim}
							%Kurstag und Gruppennummer, zb. Fr/5
			\begin{flushleft}
			\textbf{\Large{Kurstag/Gruppe:}} \Large{FR/1}
			\end{flushleft}

\begin{verbatim}






\end{verbatim}
							%Name des Betreuers, das Praktikum betreute.
			\begin{flushleft}
			\LARGE{\textbf{Betreuer:\Large{Puchegger oder Fuith}}}		
			\end{flushleft}
			
%\begin{figure}
%\includegraphics[angle=-90]{}
%\end{figure}
\section{Aufgabenstellung}
An Kondensatoren und Spulen in verschiedenen Serienschaltungen sollen die Effekte eines Schwingkreises untersucht werden, wie das Abklingen, das durch Energieverlust an Leitungswärme entsteht, oder die Effekte, die durch anschließen eines zweiten Schwingkreises entstehen: Erzwungene Schwingungen und Schwebung.
\section{Theorie}
Ein elektrischer Schwingkreis aus einem Kondensator mit einer Kapazität C und einer Spule mit einer Induktivität L in einem Stromkreis mit der Spannung U lässt sich mit der gleichen Differentialgleichung beschreiben wie ein mechanischer Oszillator:

\begin{equation}
\label{HarmOsz}
L\frac{d^2q}{dt^2}+RI-\frac{q}{C}=0
\end{equation}
\section{Aufbau}

\section{Durchführung}

\section{Ergebnisse}
\subsection{Freie Schwingungen}
Ohne Dämpfung:

11 Maxima, 2.000ms
T=200$\mu$s

From x = 200 to x = 1,800
B (y-intercept) = 3.92675897863831 +/- 0.00339724781345603
A (slope) = -0.0002725267804781 +/- 3.01853389465532e-06
--------------------------------------------------------------------------------------
Chi^2/doF = 2.18677124956392e-05
R^2 = 0.99914197688524

Mit Dämpfung:

immer Condensator C2 verwendet; R: 2,2 +- 10\% Ohm 

t=200micros

\begin{figure}
\includegraphics[angle=-90]{}
\end{figure}

\subsection{Erzwungene Schwingungen}
ungedämpft:
Resonanzfrequenz 5.00 kHz am FreqGen, laut Oszi 5.00548 kHz
Amplitude 496mV
5.43 80
5.22 112
5.11 224
5.07 296
5.05 368
5.04 424
5.00 496
4.98 440
4.96 	352
4.91 192
4.75 96
4.30 40
gedämpft:
scidavis

\subsection{Phasenverschiebung}
Bei Dämpfung (möglw zu verschmeissen:)
bei 1.2 $\omega_{max}$: 0
bei $\omega_{max}$: Pi/2
bei 0.8 $\omega_{max}$: Pi

Ohne Dämpfung:

bei 1.2 $\omega_{max}$: 0
bei $\omega_{max}$: Pi/2
bei 0.8 $\omega_{max}$: Pi


\subsection{Gekoppelte Schwingkreise}
19.73Hz Triggerfrequenz, Rechteck

Ein Bild mit ausgeschaltetem Kopplungskondensator

3microFarad

1.449kHz $\omega_S$

bzw 1.471 kHz $\omega_S$ an zweiter Spule

$\omega_M$ = 5.882khZ

bzw $\omega_M$ = 5.882kHz

6microFarad

$\omega_S$= 757.6Hz

$\omega_M$= 5.263kHz

\subsection{Eigenfrequenzen}
Bei 6 MicroFarad:
CH1 Oszi hat eine niedrigere Eigenfrequenz
5002 Hz
Ch2:
5021 Hz

bei 3 Microfarad:

Ch1:5000Hz
Ch2:5027Hz


\section{Diskussion}		
																								
\end{document}
