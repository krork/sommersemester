\documentclass[12pt,a4paper,twopage]{article}
\usepackage[utf8]{inputenc}
\usepackage[a4paper,margin=1cm,footskip=.5cm]{geometry}
\usepackage{multicol}
\usepackage{amsmath}
\usepackage{float}
\usepackage{epsfig,graphicx}
\usepackage{xcolor,import}
\usepackage{subcaption}
\usepackage[font=small,labelfont=bf]{caption}
\usepackage{siunitx}
\usepackage[german]{babel}
\usepackage{textcomp}
\usepackage{mathtools}
\linespread{1.1}
\usepackage{parskip}
\setlength{\parindent}{12pt}

\begin{document}


\thispagestyle{empty}
			\begin{center}
			\Large{Fakultät für Physik}\\
			\end{center}
\begin{verbatim}


\end{verbatim}
							%Eintrag des Wintersemesters
			\begin{center}
			\textbf{\LARGE SOMMERSEMESTER 2015}
			\end{center}
\begin{verbatim}


\end{verbatim}
			\begin{center}
			\textbf{\LARGE{Physikalisches Praktikum II}}
			\end{center}
\begin{verbatim}




\end{verbatim}

			\begin{center}
			\textbf{\LARGE{PROTOKOLL}}
			\end{center}
			
\begin{verbatim}





\end{verbatim}

			\begin{flushleft}
			\textbf{\Large{Experiment (Nr., Titel):}}\\
			PS9, Heißluftmotor - Stirlingprozess
							%Experiment Nr. und Titel statt den Punkten eintragen
			\LARGE{}	
			\end{flushleft}

\begin{verbatim}

\end{verbatim}	
							%Eintragen des Abgabedatums, oder des Erstelldatums des Protokolls
			\begin{flushleft}
			\textbf{\Large{Datum:}} \Large{29.5.2015}
			\end{flushleft}
			
\begin{verbatim}
\end{verbatim}
							%Namen der Protokollschreiber
		\begin{flushleft}
			\textbf{\Large{Bachleitner Veronika, Grafendorfer Erik}} 
			\end{flushleft}

\begin{verbatim}


\end{verbatim}
							%Kurstag und Gruppennummer, zb. Fr/5
			\begin{flushleft}
			\textbf{\Large{Kurstag/Gruppe:}} \Large{FR/1}
			\end{flushleft}

\begin{verbatim}






\end{verbatim}
							%Name des Betreuers, das Praktikum betreute.
			\begin{flushleft}
			\LARGE{\textbf{Betreuer:\Large{ }}}		
			\end{flushleft}
			
\section{Aufgabenstellung}

\section{Theorie}
\subsection{Allgemeine Grundlagen}


\section{Aufbau}

\section{Durchführung}
Die Wärmebildkamera hat eine Auflösung von 120x160 Pixel.

\section{Ergebnisse}
\subsection{Wärmeleitfähigkeit von Metallen}
\subsubsection{Nichtstationärer Wärmestrom}
Wir verwenden zur Bestimmung der Wärmeleitfähigkeit folgende Gleichung:
$$T(x,t)=(T_0 - T_{max} \cdot erf\left(\frac{1}{\sqrt{4\chi t}}x\right) + T_{max}$$
QTI-Plot verwendet als Temperaturen $T_0=24.50^\circ C$ und $T_{max}=29.75^\circ C$.\\
Als Parameter $C=\frac{1}{\sqrt{4\chi t}}$ erhalten wir
$$C=22 \pm 5$$
Damit können wir uns $\chi$ und daraus die Wärmeleitfähigkeit $\lambda$ berechnen.
$$\chi = \frac{1}{4C^2 t}=\frac{\lambda}{\rho c}$$
Wo die Dichte $\rho = (7480 \pm 350)kg/m^3$ beträgt, die Wärmekapazität $c=(454,6 \pm 0.7)Jkg^{-1}K^{-1}$ und die Zeit $t=42s$.\\
Die Wärmeleitfähigkeit ist also:
$$\boxed{\lambda = 41.82 W/m K}$$
%UNSICHERHEIT!!!
\\
\subsubsection{Stationärer Wärmestrom}
Der zugeführte Wärmestrom hat den Wert:
$$\Phi=(12.91 \pm 0.01)V \cdot (0.039 \pm 0.001)A=0.50 \pm 0.01 J/s$$
wo die Unsicherheit berechnet wird mittels:\\
Wert $\cdot$ relative Unsicherheit $= (12.92 \cdot 0.039) \cdot \sqrt{\frac{0.01}{12.92}^2 + \frac{0.001}{0.039}^2}$\\


Querschnittsfläche: $A=(10.2 \cdot 4.4)=(44.9 \pm 1.5)mm^2$\\
Steigung: $\lambda \cdot A = (-110 \pm 2)$\\
$$\boxed{\lambda=105.33 W/m K}$$

\subsection{Isolatoren}
Größe der Platten:\\
Dicke der Referenzplatte $d_{Ref}=(10.00 \pm 0.01)mm$.\\
Fläche der Referenzplatte: $A_{Ref}=15.05^2=(226.5 \pm 0.3)mm^2$\\
\\
Dicke der Messplatte $d_{Ref}=(10.02 \pm 0.01)mm$.\\
Fläche der Messplatte: $A_{Ref}=15.03 \cdot 14.95=(224.7 \pm 0.3)mm^2$\\
\\
Die Platten haben also etwa gleiche Fläche und Dicke, weswegen wir diese im Folgenden vernachlässigen können.\\

\begin{table}
\begin{center}
\begin{tabular}{|l|l|l|l|l|}
\hline
Zeit (min) & $T_{Ofen}$ $(^\circ C)$ & $T_{Mess}$ $(^\circ C)
$& $T_{Ref,1}$ $(^\circ C)$ & $T_{Ref,2}$ $(^\circ C)$\\
\hline
0 & 48.1 & 25.7 & 36.9 & 29.6\\
30 & 56.9 & 28.4 & 43.6 & 34.4\\
60 & 61.0 & 30.8 & 48.3 & 38.8\\
90 & 63.9 & 31.9 & 51.0 & 41.4\\
120 & 65.9 & 32.7 & 53.0 & 42.5\\
150 & 66.9 & 33.1 & 54.2 & 44.0\\
\hline
\end{tabular}
\caption{Gemessene Temperaturen in Zeitschritten von 30 Minuten}
\end{center}
\end{table}

\section{Diskussion}
																								
\end{document}
