\documentclass{article}
\usepackage[utf8]{inputenc}
\usepackage{multicol}
\usepackage{amsmath}
\usepackage{float}
\usepackage{epsfig,graphicx}
\usepackage{xcolor,import}
\usepackage{subcaption}
\usepackage[font=small,labelfont=bf]{caption}
\usepackage{siunitx}
\usepackage[german]{babel}
\usepackage{textcomp}
\usepackage{mathtools}

\begin{document}


\thispagestyle{empty}
			\begin{center}
			\Large{Fakultät für Physik}\\
			\end{center}
\begin{verbatim}


\end{verbatim}
							%Eintrag des Wintersemesters
			\begin{center}
			\textbf{\LARGE SOMMERSEMESTER 2015}
			\end{center}
\begin{verbatim}


\end{verbatim}
			\begin{center}
			\textbf{\LARGE{Physikalisches Praktikum II}}
			\end{center}
\begin{verbatim}




\end{verbatim}

			\begin{center}
			\textbf{\LARGE{PROTOKOLL}}
			\end{center}
			
\begin{verbatim}





\end{verbatim}

			\begin{flushleft}
			\textbf{\Large{Experiment 3: Radioaktivität}}\\
							%Experiment Nr. und Titel statt den Punkten eintragen
			\LARGE{}	
			\end{flushleft}

\begin{verbatim}

\end{verbatim}	
							%Eintragen des Abgabedatums, oder des Erstelldatums des Protokolls
			\begin{flushleft}
			\textbf{\Large{Datum:}} \Large{27.03.2015}
			\end{flushleft}
			
\begin{verbatim}
\end{verbatim}
							%Namen der Protokollschreiber
		\begin{flushleft}
			\textbf{\Large{Bachleitner Veronika, Grafendorfer Erik}} 
			\end{flushleft}

\begin{verbatim}


\end{verbatim}
							%Kurstag und Gruppennummer, zb. Fr/5
			\begin{flushleft}
			\textbf{\Large{Kurstag/Gruppe:}} \Large{FR/1}
			\end{flushleft}

\begin{verbatim}






\end{verbatim}
							%Name des Betreuers, das Praktikum betreute.
			\begin{flushleft}
			\LARGE{\textbf{Betreuer:\Large{ }}}		
			\end{flushleft}
			
\section{Aufgabenstellung}

\section{Theorie}
\subsection{}

\section{Aufbau}

\section{Durchführung}

\section{Ergebnisse}


\subsection{Körpereigene Radioaktivität}
Wir haben $140g$ Kalium im Körper (ausgehend von $70kg$ Körpergewicht).
Davon sind
$$
140g\cdot 0.000117=1.638\cdot10^{-2}g \\
$$
Kalium 40.\\
Kalium40 hat $40\frac{g}{mol}$. Also haben wir \\
$$4.095\cdot10^{-4}mol$$ Kalium 40 im Körper.\\
\\
Wenn wir dies mit der Avogadro-Konstante multiplizieren erhalten wir:\\
$$2.46607\cdot 10^{20}$$ Atome Kalium 40.\\
\\
Mit einer Halbwertszeit von $T_{\frac{1}{2}}= 1.28 \cdot 10^{9}a=4.039\cdot 10^{16}s$ \\
bekommen wir die Lebensdauer $$\tau=\frac{T_{1}{2}}{\log{2}}=\lambda^{-1} \Rightarrow \lambda=1.7159 \cdot 10^{-17}$$
\\
Daraus gelangen wir schließlich zu dem Ergebnis, dass die körpereigene Aktivität folgenden Wert hat:
$$A=N\cdot \lambda = 4.2317\cdot 10^{3} \frac{Ereignisse}{Sekunde}$$
\subsection{Szintillationszähler}
\subsubsection{Natrium 22}
Wir sehen Peaks bei $(513.10 \pm 5.00)keV$ und $(1275.97 \pm 27.80)keV$.

%Bild mit den Peaks einfügen

\subsubsection{Unbekannte Probe}
Wir sehen Peaks bei $(1181.16 \pm 7.90)keV$ und $(1333.78 \pm 14.29)keV$.\\
Aufgrund der gemessenen Peaks stellen wir fest, dass es sich bei der unbekannten Probe um \textbf{Cobalt 60} handeln muss.

%Bild mit den Peaks einfügen

\subsection{Abstandsabhängigkeit}
%N4:131.44mm : 187 :
%N5:127.30mm : 178 :
%N6:123.22 : 199 :
%N7:115.00 : 209 :
%N8:98.50 :260 :
%N9:65.7 : 479 Ereignisse:
\begin{tabular}{|r|l|}
\hline
Abstand (mm) & Zählrate\\
\hline
127.30 & 178\\
123.22 & 199\\
115.00 & 209\\
98.50 & 260\\
65.72 & 479\\
\hline
\end{tabular}

%Fit von dieser Tabelle einfügen

\subsection{Nebelkammer}
\subsection{Geiger-Müller-Zählrohr}

\section{Diskussion}		
																								
\end{document}
